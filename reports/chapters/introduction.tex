\chapter{Introduction}
\label{ch:into} % This how you label a chapter and the key (e.g., ch:into) will be used to refer this chapter ``Introduction'' later in the report. 
% the key ``ch:into'' can be used with command \ref{ch:intor} to refere this Chapter.

\begin{itemize}
    \item A complete blood cell count (CBC) is vital for assessing health, comprising red blood cells (RBCs), white blood cells (WBCs), and platelets. Manual counting methods are time-consuming and error-prone, requiring automation. Machine learning, particularly deep learning, offers robust solutions across medical applications. Applying deep learning to identify and count blood cells in smear images presents a promising avenue for accurate and efficient analysis, revolutionizing medical diagnostics. Previous models like YOLOv5 and YOLOX have pushed the boundaries of object detection with improved speed and accuracy. YOLOv5 introduced innovations like ConvBNLeakyReLU and EfficientNet-inspired components. YOLOX further enhanced performance with methods like Cross Stage Partial Network plus CBS. While these models have advanced the field, they may face limitations in handling small objects or complex scenes
       

%%%%%%%%%%%%%%%%%%%%%%%%%%%%%%%%%%%%%%%%%%%%%%%%%%%%%%%%%%%%%%%%%%%%%%%%%%%%%%%%%%%
\section{Background}
\label{sec:into_back}
The utilization of image-based methods for disease detection and diagnosis has gained significant attention in recent years. This project focuses on the development of an automated system for the identification and counting of blood cells, leveraging advanced image processing and machine learning techniques.

%%%%%%%%%%%%%%%%%%%%%%%%%%%%%%%%%%%%%%%%%%%%%%%%%%%%%%%%%%%%%%%%%%%%%%%%%%%%%%%%%%%
\section{Problem statement}
\label{sec:intro_prob_art}
The accurate identification and characterization of blood cells, particularly red blood cells (RBCs) and white blood cells (WBCs), pose significant challenges in traditional medical diagnostics. Manual methods for blood cell analysis are often labor-intensive, time-consuming, and prone to human error, leading to variability in results.
%%%%%%%%%%%%%%%%%%%%%%%%%%%%%%%%%%%%%%%%%%%%%%%%%%%%%%%%%%%%%%%%%%%%%%%%%%%%%%%%%%%
\section{Aims and objectives}
\label{sec:intro_aims_obj}
Aims: This project aims to develop an automated system for blood cell analysis to improve accuracy, efficiency, and reliability in disease diagnosis.

Objectives: The specific objectives of this project include: 

*Exploring existing methodologies for automated blood cell analysis.

*Identifying limitations in current approaches and proposing innovative solutions. 

*Designing and implementing a novel solution approach combining image processing and machine learning techniques. 
*Evaluating the performance of the developed system and comparing it with existing methods.




%%%%%%%%%%%%%%%%%%%%%%%%%%%%%%%%%%%%%%%%%%%%%%%%%%%%%%%%%%%%%%%%%%%%%%%%%%%%%%%%%%%
\section{Solution approach}
\label{sec:intro_sol} % label of Org section
Briefly, the solution approach involves leveraging advanced algorithms and methodologies from image processing and machine learning domains. This includes preprocessing techniques for image enhancement, feature extraction methods, and the implementation of machine learning models for classification and counting of blood cells.



%%%%%%%%%%%%%%%%%%%%%%%%%%%%%%%%%%%%%%%%%%%%%%%%%%%%%%%%%%%%%%%%%%%%%%%%%%%%%%%%%%%
\section{Summary of contributions and achievements} %  use this section 
\label{sec:intro_sum_results} % label of summary of results
 This study makes several significant contributions to the field of automated blood cell analysis. By addressing key challenges and proposing innovative solutions, we aim to: Improve the accuracy and reliability of blood cell identification and counting. Streamline the analysis process, thereby reducing time and labor requirements. Enhance the diagnostic capabilities of healthcare professionals, leading to improved patient outcomes.


%%%%%%%%%%%%%%%%%%%%%%%%%%%%%%%%%%%%%%%%%%%%%%%%%%%%%%%%%%%%%%%%%%%%%%%%%%%%%%%%%%%
\section{Organization of the report} %  use this section
\label{sec:intro_org} % label of Org section
The report is organized into several sections for clarity and coherence. It begins with an Introduction, covering background, problem statement, aims, objectives, and solution approach. Following this, the Literature Review discusses pertinent sources and citation practices. Methodology outlines the research methodology adopted. Results present the findings obtained from the study. Discussion and Analysis critically analyze the results, highlighting their significance and limitations. Conclusions summarize the key findings and suggest avenues for future research. Finally, Appendices include supplementary materials such as data tables or additional information for interested readers. This structure aims to guide readers through the report's content efficiently and comprehensively.

