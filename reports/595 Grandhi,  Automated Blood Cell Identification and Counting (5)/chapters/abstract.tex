%Two resources useful for abstract writing.
% Guidance of how to write an abstract/summary provided by Nature: https://cbs.umn.edu/sites/cbs.umn.edu/files/public/downloads/Annotated_Nature_abstract.pdf %https://writingcenter.gmu.edu/guides/writing-an-abstract
\chapter*{\center \Large  Abstract}
%%%%%%%%%%%%%%%%%%%%%%%%%%%%%%%%%%%%%%
% Replace all text with your text
%%%%%%%%%%%%%%%%%%%%%%%%%%%%%%%%%%%

In our cutting-edge approach to automate the identification and counting of three blood cell 
types, we employ a sophisticated fusion of deep learning and advanced image processing 
techniques, particularly focusing on object detection. Traditional complete blood cell counts 
necessitate laborious manual counting using a haemocytometer, involving intricate laboratory 
equipment and chemical compounds. This antiquated method is both time-consuming and burdensome. 
Our innovative solution utilizes Convolutional Neural Networks (CNNs) for intricate feature 
extraction from microscopic blood sample images. Incorporating state-of-the-art object 
detection algorithms, such as YOLO (You Only Look Once) or Faster R-CNN (Region-based 
Convolutional Neural Network), our system precisely identifies and localizes individual blood 
cells, overcoming the limitations of manual counting. Image processing techniques, including 
contrast enhancement and morphological operations, are strategically applied to optimize image 
quality and facilitate accurate object segmentation. This synergistic blend of deep learning 
and image processing not only expedites the diagnostic process but also significantly improves 
the accuracy and efficiency of blood cell identification and counting. By automating this 
intricate task, our approach aims to revolutionize medical diagnostics, providing healthcare 
professionals with a rapid and reliable tool for comprehensive blood cell analysis.

%%%%%%%%%%%%%%%%%%%%%%%%%%%%%%%%%%%%%%%%%%%%%%%%%%%%%%%%%%%%%%%%%%%%%%%%%s
~\\[1cm]
\noindent % Provide your key words
\textbf{Keywords:} a maximum of five keywords/keyphrase separated by commas

\vfill
\noindent


